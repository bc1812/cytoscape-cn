Cytoscape~可以接受很多命令行参数,包括指定网络文件、属性文件和会话文件。下面的内容是用
Cytoscape~的“-h”或“--help”运行参数得到的:
\begin{verbatim}
usage: java -Xmx512M -jar cytoscape.jar [OPTIONS]
 -h,--help               Print this message.
 -v,--version            Print the version number.
 -s,--session <file>     Load a cytoscape session (.cys) file.
 -N,--network <file>     Load a network file (any format).
 -e,--edge-attrs <file>  Load an edge attributes file (edge attribute format).
 -n,--node-attrs <file>  Load a node attributes file (node attribute format).
 -m,--matrix <file>      Load a node attribute matrix file (table).
 -p,--plugin <file>      Load a plugin jar file, directory of jar files,
                         plugin class name, or plugin jar URL.
 -P,--props <file>       Load cytoscape properties file (Java properties
                         format) or individual property: -P name=value.
 -V,--vizmap <file>      Load vizmap properties file (Java properties format).
\end{verbatim}

在指定文件时,可以是本地文件的路径,也可以是URL。例如,可以这样指定本地的网络文件(假
设myNet.sif位于当前工作目录):cytoscape.sh -N myNet.sif。还可以用URL指定网络:
cytoscape.sh -N http://example.com/myNet.sif。
\begin{description}
\item[参数] 用途
\item[-h,--help] 显示上面已经展示过的帮助信息,然后退出程序。
\item[-v,--version] 显示~Cytoscape~的版本号,然后退出程序。

\item[-s,--session $<$file$>$] 这个选项用于加载指定的会话文件。
由于会话文件只能在特定的时间加载,所以这个选项在命令行中只能使用一次。要给这个选项指
定一个.cys~的~Cytoscape~会话文件。不过,cys~后缀并不是必须的。
\item[-N,--network $<$file$>$]
这个选项用来加载各种类型的网络文件,包括~SIG~、~GML~和~XGMML的各种格式的文件都可以通过- N
选项加载到~Cytoscape~中。在一个命令中可以同时加载多个网络。
\item[-e,--edge-attrs $<$file$>$] 这个选项用来指定边属性文件。
在一个命令中可以同指定多个边属性文件。
\item[-n,--node-attrs $<$file$>$] 这个选项用于指定节点属性文件。在一个命令中可以同指定多个边属性文件。
\item[-m,--matrix $<$file$>$]这个选项用于指定数据矩阵文件。
如果是用于生物网络分析,这个数据矩阵的内容就是基因表达数据。所有的数据矩阵文件都会以
节点属性的形式读入Cytoscape。在一个命令中可以同指定多个数据矩阵文件。
\item[-p,--plugin $<$file$>$]这个选项用于将指定的Cytoscape插件(.jar)文件加载到
Cytoscape中。老版本中的“资源插件选项(resource plugin option)”已经合并到了这个选项
中\footnote{这句话的翻译不确定,请有了解老版本Cytoscape的同学指正}。如果插件位于
Cytoscape的CLASSPATH路径中,还可以用类名称来制定插件。例如,假设能在~CLASSPATH~中找
到名为~MyPlugin~的类,那么就可以这样指定插件:cytoscape.sh -p MyPlugin.class。最后
一种指定插件的方面就是指定一个文本文件,在这个文本文件中列出所有需要加载的插件的~jar~文件名。 
\item[-P,--props
$<$file$>$]这个选项可以用来指定~Cytoscape~的各项属性。可以用过属性文件(格式是标准的
~Java~属性格式),也可以用等号设置单个属性的值。例如,指定缺省的物种:
cytoscape.sh -P defaultSpeciesName=Human。如果属性中有空格,加上引号就可以了,例如:
cytoscape.sh -P "defaultSpeciesName=Homo Sapiens"。
这个选项可以用来代替以前的-noCanonicalization,-species和-bioDataServer选项。现在的
命令可以写成这样的:cytoscape.sh -P defaultSpeciesName=Human -P noCanonicalization=true
-P bioDataServer=myServer.
\item[-V,--vizmap $<$file$>$] 这个选项用来指定可视化属性文件。
\end{description}

上面介绍的这些属性(包括插件)都可以在正在运行的Cytoscape的GUI中加载。