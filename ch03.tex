Cytoscape~可以接受很多命令行参数,包括指定网络文件、属性文件和会话文件。下面的内容是用
Cytoscape~的“-h”或“--help”运行参数得到的:
\begin{verbatim}
usage: java -Xmx512M -jar cytoscape.jar [OPTIONS]
 -h,--help               Print this message.
 -v,--version            Print the version number.
 -s,--session <file>     Load a cytoscape session (.cys) file.
 -N,--network <file>     Load a network file (any format).
 -e,--edge-attrs <file>  Load an edge attributes file (edge attribute format).
 -n,--node-attrs <file>  Load a node attributes file (node attribute format).
 -m,--matrix <file>      Load a node attribute matrix file (table).
 -p,--plugin <file>      Load a plugin jar file, directory of jar files,
                         plugin class name, or plugin jar URL.
 -P,--props <file>       Load cytoscape properties file (Java properties
                         format) or individual property: -P name=value.
 -V,--vizmap <file>      Load vizmap properties file (Java properties format).
\end{verbatim}
Any file specified for an option may be specified as either a path or as a URL.
For example you can specify a network as a file (assuming that myNet.sif exists
in the current working directory): cytoscape.sh -N myNet.sif. Or you can specify
a network as a URL: cytoscape.sh -N http://example.com/myNet.sif.

\begin{table}
\begin{tabular}{|l|l|}
\hline
Argument & Description \\
\hline
-h,--help & This flag generates the help output you see above and exits.\\
\hline
-v,--version & This flag prints the version number of Cytoscape and exits.\\
\hline
-s,--session <file> & This option specifies a session file to be loaded. Since only one session file
can be loaded at a given time, this option may only specified once on a given
command line. The option expects a .cys Cytoscape session file. It is customary,
although not necessary, for session file names to contain the .cys extension.\\

-N,--network <file>
    

This option is used to load all types of network files. SIF, GML, and XGMML files
can all be loaded using the -N option. You can specify as many networks as
desired on a single command line.

-e,--edge-attrs <file>
    

This option specifies an edge attributes file. You may specify as many edge
attribute files as desired on a single command line.

-n,--node-attrs <file>
    

This option specifies a node attributes file. You may specify as many node
attribute files as desired on a single command line.

-m,--matrix <file>
    

This option specifies a data matrix file. In a biological context, the data
matrix consists of expression data. All data matrix files are read into node
attributes. You may specify as many data matrix files as desired on a single
command line.

-p,--plugin <file>
    

This option specifies a cytoscape plugin (.jar) file to be loaded by Cytoscape.
This option also subsumes the previous "resource plugin option". You may specify
a class name that identifies your plugin and the plugin will be loaded if the
plugin is in Cytoscape's CLASSPATH. For example, assuming that the class MyPlugin
can be found in the CLASSPATH, you could specify the plugin like this:
cytoscape.sh -p MyPlugin.class. A final means of specifying plugins is to specify
a file name whose contents contain a list of plugin jar files.

-P,--props <file>
    

This option specifies Cytoscape properties. Properties can be specified either as
a properties file (in Java's standard properties format), or as individual
properties. To specify individual properties, you must specify the property name
followed by the property value where the name and value are separated by the '='
sign. For example to specify the defaultSpeciesName: cytoscape.sh -P
defaultSpeciesName=Human. If you would like to include spaces in your property,
simply enclose the name and value in quotation marks: cytoscape.sh -P
"defaultSpeciesName=Homo Sapiens". The property option subsumes previous options
-noCanonicalization, -species, and -bioDataServer. Now it would look like:
cytoscape.sh -P defaultSpeciesName=Human -P noCanonicalization=true -P
bioDataServer=myServer.

-V,--vizmap <file>
    

This option specifies a visual properties file.
\end{tabular}
\end{table}
All options described above (including plugins) can be loaded from the GUI once
Cytoscape is running.