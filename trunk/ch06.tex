Cytoscape~可以读取以下格式的网络或路径文件:
\begin{itemize}
\item Simple interaction file (SIF or .sif format) 
\item Graph Markup Language (GML or .gml format) 
\item XGMML (extensible graph markup and modelling language). 
\item SBML 
\item BioPAX 
\item PSI-MI Level 1 and 2.5 
\item Delimited text 
\item Excel Workbook (.xls) 
\end{itemize}

SIF~格式的文件只有节点和相互作用,而其它的格式都可以存储网络布局信息,还可以跟其它的网络软件和数据源交换数据。SIF~文件通常用于在新建网络时导入相互作用,因为用文本编辑器和电子表格软件能很方便的创建这种格式的文件。在导入了相互作用,应用了某种网络布局后,就可以将网络存为GML或XGMML格式,从而能去其它系统交换数据。所有的这些格式(Excel例外)都是文本文件,用普通的文本编辑器就能编辑和查看这些文件。


\section{SIF~格式}
这种简单的格式可以很方便地用于从相互作用列表构建网络。利用这种格式,还能很方便的把小网络组合在一起,或是在现有的数据中添加新的相互作用。但这种格式的缺点也是显而易见的,其中不包含布局信息,这使得Cytoscape不得不在每次加载网络时都重新计算网络的布局。

SIF~文件中的每行都由起点、相互作用类型(或边的类型)和一个或若干个重点构成。
\begin{verbatim}
nodeA <relationship type> nodeB
nodeC <relationship type> nodeA
nodeD <relationship type> nodeE nodeF nodeB
nodeG
...
nodeY <relationship type> nodeZ
\end{verbatim}

下面是一个具体的例子:
 \begin{verbatim}
node1 typeA node2
node2 typeB node3 node4 node5
node0
\end{verbatim}

第一行是两个节点,node1和node2,以及两点间的typeA型的相互关系。第二行加入了三个新节点,node3~、\linebreak node4~和~node5,这一行中的node2跟第一行中的node2是同一个节点。第二行还设定了三条起点相同且类型也相同的相互作用。第三行说明了如何引入孤立的节点。 This form is not needed for nodes that do have relationships, since the specification of the relationship implicitly identifies the nodes as well. 

重复的条目被忽略。两个点之间的多条相互作用必须是不同的类型。例如,下面的node1和node2之间就有两条边,但类型分别是xx和yy。
\begin{verbatim}
node1 xx node2
node1 xx node2
node1 yy node2
\end{verbatim}

节点的自相互作用是允许的:
\begin{verbatim}
node1 xx node1
\end{verbatim}

Cytoscape中每个点和每条边都有标识符,通常在节点和边的属性数据中都有。节点的名称必须是独一无二的,名称相同的节点会被视为同一个节点。节点的名称缺省情况下就是文件中的名称,除非用Visual Mapper映射到了其他字符串。详见``视觉风格''一章。边的名称则是由边的起点和终点,加上相互作用的类型构成的,例如:sourceName (edgeType) targetName。


$<$relationship type$>$标签可以是任何字符串。完整的单词或是单词的组合都可以用来定义相互作用的类型,例如:geneFusion、cogInference、pullsDown、activates、degrades、inactivates~、~inhibits~、~phosphorylates~、~upRegulates~等等。 

在系统生物学领域,常见的相互作用类型有:
\begin{enumerate}
\item  pp .................. protein $\rightarrow$ protein interaction
\item  pd .................. protein $\rightarrow$ DNA   
  (例如,转录因子跟调控基因的上游结合。)
\end{enumerate}

还有一些相对少见的相互作用类型:
\begin{enumerate}
\item  pr .................. protein $\rightarrow$ reaction
\item  rc .................. reaction $\rightarrow$ compound
\item  cr .................. compound $\rightarrow$ reaction
\item  gl .................. genetic lethal relationship
\item  pm .................. protein-metabolite interaction
\item  mp .................. metabolite-protein interaction
\end{enumerate}


\textbf{分隔符}

在简单的相互作用文件格式中,空白(空格或制表符)用来分割名称。但在有些情况,节点名称或是边的类型中会含有空格。Cytoscape对分隔符的处理原则是这样的:如果文件中有制表符,那么就用制表符分割不同的字段,而空格则被视为字段中的内容。如果文件中没有制表符,那么所有的空格就都是分隔符(也就是中字段中不会有空格)。

如果在导入网路后,发现网络中没有边,而且点的名称也有问题,那很有可能是因为文件中存在制表符,使得Cytoscape在导入网络时判断错误。另一方面,如果网络中节点的名称只是正确名称的一部分,那就很有可能是错误地将空格作为了分隔符,实际上应该用制表符。

用简单的相互作用的形式存放网络的文件的后缀是.sif,Cytoscape能识别出文件夹中所含有的这类文件。

\section{GML~格式}
跟SIF格式不同,GML是一种富图格式语言(rich graph format language),很多网络可视化软件包都支持这种格式。在\url{http://www.infosun.fmi.uni-passau.de/Graphlet/GML/}上可以找到该格式的具体说明。

通常都不必直接修改GML文件的内容。当SIF格式的网络导入并实施布局后,就可以以GML的形式保存和加载网络。GML文件中的视觉风格在加载GML文件后,会保存为名为Filename.style的视觉风格。

\section{XGMML~格式}
XGMML是GML的XML扩展版,它是基于GML定义的。除了网络数据,XGMML还包含节点、边和网络的属性。XGMML文件的规范在\url{http://www.cs.rpi.edu/~puninj/XGMML/}上。

由于XGMML继承了XML文件的灵活性,所以XGMML比GML更受欢迎。如果不确定应该用哪种格式,那就选择XGMML吧。

\section{系统生物学标记语言(Systems Biology Markup Language)}
系统生物学标记语言(Systems Biology Markup language)是一种用于描述生化网络的XML格式。SBML文件格式规范:\url{http://sbml.org/documents/}
 
\section{BioPAX (Biological PAthways eXchange)格式}

BioPAX是一种用于交换生物路径数据的OWL(Web Ontology Language)文档。该格式的完整说明文档:\url{http://www.biopax.org/index.html}

\section{PSI-MI~格式}
PSI-MI格式一种用于描述蛋白质相互作用及有关数据的XML格式。PSI-MI XML的格式规范:
\url{http://psidev.sourceforge.net/mi/xml/doc/user/}。

\section{纯文本表格和Excel表格}

Cytoscape~为微软的Excel文件(.xls)和纯文本表格提供了原生支持。可以用这些表格存储网络数据和边的属性。在导入文件的过程中,用户可以指定哪一列是源节点,哪一列是终点,以及相互作用的类型、边属性等等。其它的一些网络分析工具,例如igraph (\url{http://cneurocvs.rmki.kfki.hu/igraph/}), 也可以将网络输出成简单的文本文件。Cytoscape~可以读取这些文本文件,并构建网络。具体信息请阅读\ref{free-format table} 节。

\centerline{\includegraphics[width=.7\textwidth]{images/huge_network_igraph.png}}

\textbf{在Cytoscape中显示根据~igraph~的Watts-Strogatz小世界模型生成的网络(含有5万个节点,25万条边)}
 
利用Table Import功能,可以读取其它软件生成的网络。
 
\section{Cytoscape~中节点的命名}

通常情况下,节点表示基因,节点间的边表示相互作用(或者是其他的生物关系)。
为了紧凑,基因也可以表示成响应的蛋白质。节点还可以用来表示化合物和生化反应等各种
东西,并不局限于基因。

如果想要将基因或蛋白跟GO注释或基因表达数据整合在一起,那各个数据文件中的基因名称
必须完全一致。
If a network of genes or proteins is to be integrated with Gene Ontology (GO)
annotation or gene expression data, the gene names must exactly match the names
specified in the other data files. We strongly encourage naming genes and
proteins by their systematic ORF name or standard accession number; common
names may be displayed on the screen for ease of interpretation, so long as
these are available to the program in the annotation directory or in a node
attribute file. Cytoscape ships with all yeast ORF-to-common name mappings in a
synonym table within the annotation/ directory. Other organisms will be
supported in the future. 

 Why do we recommend using standard gene names? All of the external data
formats recognized by Cytoscape provide data associated with particular names
of particular objects. For example, a network of protein-protein interactions
would list the names of the proteins, and the attribute and expression data
would likewise be indexed by the name of the object. 

 The problem is in connecting data from different data sources that don't
necessarily use the same name for the same object. For example, genes are
commonly referred to by different names, including a formal ``location on the
chromosome'' identifier and one or more common names that are used by ordinary
researchers when talking about that gene. Additionally, database identifiers
from every database where the gene is stored may be used to refer to a gene
(e.g. protein accession numbers from Swiss-Prot). If one data source uses the
formal name while a different data source used a common name or identifier,
then Cytoscape must figure out that these two different names really refer to
the same biological entity. 

 Cytoscape has two strategies for dealing with this naming issue, one simple
and one more complex. The simple strategy is to assume that every data source
uses the same set of names for every object. If this is the case, then
Cytoscape can easily connect all of the different data sources. 

 To handle data sources with different sets of names, as is usually the case
when manually integrating gene information from different sources, Cytoscape
needs a data server that provides synonym information (see the chapter on
Annotation). A synonym table gives a canonical name for each object in a given
organism and one or more recognized synonyms for that object. Note that the
synonym table itself defines which set of names are the ``canonical'' names.
For example, in budding yeast, the ORF names are commonly used as the canonical
names. 

 If a synonym server is available, then by default Cytoscape will convert every
name that appears in a data file to the associated canonical name. Unrecognized
names will not be changed. This conversion of names to a common set allows
Cytoscape to connect the genes present in different data sources, even if they
have different names -- long as those names are recognized by the
synonym server. 

 For this to work, Cytoscape must also be provided with the species to which
the objects belong, since the data server requires the species in order to
uniquely identify the object referred to by a particular name. This is usually
done in Cytoscape by specifying the species name on the command line with the
-P option (cytoscape.sh -P ``defaultSpeciesName=Saccharomyces
cerevisiae'') or by editing the properties (under Edit $\rightarrow$ Preferences
$\rightarrow$ Properties\ldots). 

 The automatic canonicalization of names can be turned off using the -P option
(cytoscape.sh -P ``canonicalizeName=false'') or by editing the properties (under
Edit $\rightarrow$ Preferences $\rightarrow$ Properties\ldots). This canonicalization
of names currently does not apply to expression data. Expression data should
use the same names as the other data sources or use the canonical names as
defined by the synonym table. 
