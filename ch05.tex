在Cytoscape中有四种创建网络的方法:
\begin{enumerate}
\item 导入预设的格式化网络文件。
\item 导入预设的未格式化的文本文件或Excel文件。
\item 从Web Service导入网络。
\item 创建一个空网络,然后手动地添加节点和边。
\end{enumerate}
\section{Import Fixed-Format Network Files}

 Network files can be specified in any of the formats described in the Supported Network Formats chapter. Networks are imported into Cytoscape through the ``Import Network'' window, which can be accessed by going to File $\rightarrow$ Import $\rightarrow$ Network (multiple file types). The network file can either be located directly on the local computer, or found on a remote computer (in which case it will be referenced with a URL). 


 \textbf{Load Networks from Local Computer}


 By default, Cytoscape loads networks from the local computer. 


 The Import Networks dialog shows a default setting of ``Data Source Type: Local,'' meaning that network files from the local computer will be available for importing. Choose the correct file by clicking on the Select button (only file types that Cytoscape recognizes will be shown), and then load the network by clicking on the Import button. Some sample network files of different types have been included in the sampleData folder in Cytoscape. 


 Network files in SIF, GML, and XGMML formats may also be loaded directly from the command line using the \^a€“N option. 


 \includegraphics[wdith=\textwidth]{images/network_import_dialog1_25.png} 


 
\textbf{Load Networks from a Remote Computer (URL import)}


 The Import Networks dialog is also capable of importing network files using a URL. To do this, set the Data Source Type to Remote and insert the appropriate URL, either manually or using URL bookmarks. Bookmarked URLs can be accessed by clicking on the arrow to the right of the text field (see the Bookmark Manager in Preferences for more details on bookmarks). Also, you can drag and drop links from web browser to the URL text box. Once a URL has been specified, click on the Import button to load the network. 


 \includegraphics[wdith=\textwidth]{images/network_import_dialog2_25.png} 


 Importing networks from URL addresses has an important caveat. Because Cytoscape determines file type primarily (not exclusively) by file extension, it can have trouble importing networks with URLs that don't end in a human readable file name. If Cytoscape does not recognize a meaningful file name and extension in the URL, it will attempt to guess the type of file based on MIME type. If the MIME type is not recognizable to any of our import handlers, then the import will fail. 


 Another issue for network import is the presence of firewalls, which can affect which files are accessible to a computer. To work around this problem, Cytoscape supports the use of proxy servers. To configure the proxy server, go to Edit $\rightarrow$ Preferences$\rightarrow$ Proxy Server... . This is further described in the Preferences chapter. 


  

\section{Import Free-Format Table Files}


 Introduced in version 2.4, Cytoscape now supports the import of networks from delimited text files and Excel workbooks using Edit $\rightarrow$ Import $\rightarrow$ Network from Table (Text/MS Excel)... . An interactive GUI allows users to specify parsing options for specified files. The screen provides a preview that shows how the file will be parsed given the current configuration. As the configuration changes, the preview updates automatically. In addition to specifying how the file will be parsed, the user must also choose the columns that represent the Source nodes, the Target nodes, and an optional edge interaction type. 


 \includegraphics[wdith=\textwidth]{images/network_table_import.png} 


 \textbf{Supported Files}


 The ``Import Network from Table'' function supports delimited text files and single-sheet Microsoft Excel Workbooks. The following is a sample table file: 


 \begin{verbatim}
source  target  interaction  boolean attribute  string attribute        floating point attribute
YJR022W YNR053C pp      TRUE    abcd12371       1.2344543
YER116C YDL013W pp      TRUE    abcd12372       1.2344543
YNL307C YAL038W pp      FALSE   abcd12373       1.2344543
YNL216W YCR012W pd      TRUE    abcd12374       1.2344543
YNL216W YGR254W pd      TRUE    abcd12375       1.2344543

\end{verbatim}



 The network table files should contain at least two columns: source nodes and target nodes. The interaction type is optional in this format. Therefore, a minimal network table looks like the following: 


 \begin{verbatim}
YJR022W YNR053C
YER116C YDL013W
YNL307C YAL038W
YNL216W YCR012W
YNL216W YGR254W

\end{verbatim}



 One row in a network table file represents an edge and its edge attributes. This means that a network file is considered a combination of network data and edge attributes. A table may contain columns that aren't meant to be edge attributes. In this case, you can choose not to import those columns by clicking on the column header in the preview window. This function is useful when importing a data table like the following (1): 
\begin{verbatim}
Unique ID A     Unique ID B     Alternative ID A        Alternative ID B        Aliases A       Aliases B       Interaction detection methods   First author surnames   Pubmed IDs      species A       species B       Interactor types        Source database Interaction ID  Interaction labels      Cross-references        Associated Files        Experiment files        Experiment labels       Different techniques    Different Pubmed articles       Different sources       Weight

7205    5747    TRIP6   PTK2    Q15654  Q05397-1        vv|HPRD Currently not available 14688263|15892868(Marcotte)     Mammalia        Homo sapiens    protein|protein HPRD|Marcotte   0       Thyroid hormone receptor interactor 6-FAK-|PTK2-TRIP6   NA(HPRD)|NA(Marcotte)   HPRD/02859_psimi.xml|other/ORIGINAL_DATA_MARCOTTE.txt   vv(HPRD/02859_psimi.xml)|HPRD(other/ORIGINAL_DATA_MARCOTTE.txt) 17651(ExptRef)|Marcotte 2       2       2       2

4174    7311    MCM5    UBA52   P33992  P62987  neighbouring_reaction   Currently not available 15608231(Reactome)      Homo sapiens    Homo sapiens    protein|protein Reactome        1       P33992-P62988   Reaction:68944<->Reaction:68946(Reactome)|Reaction:68946<->Reaction:68944(Reactome)     other/ORIGINAL_DATA_MARCOTTE.txt        neighbouring_reaction(other/REACTOMEhomo_sapiens.interactions.txt)      Reactome        1       1       1       1

7040    7040    TGFB1   TGFB1   P01137  P01137  nmr: nuclear magnetic resonance Currently not available 8679613 Homo sapiens    Homo sapiens    protein|protein BIND    2       TGFB1-TGFB1-    72085(BIND)     BIND/bind_taxid9606.1.psi.xml   nmr: nuclear magnetic resonance(BIND/bind_taxid9606.1.psi.xml)  NotAvailable    1       1       1       1

\end{verbatim}


 This data file is a tab-delimited text and contains network data (interactions), edge attributes, and node attributes. To import network and edge attributes from this table, you need to choose Unique ID A as source, Unique ID B as target, and Interactor types as interaction type. Then you need to turn off columns used for node attributes (Alternative ID A, species B, etc.). Other columns can be imported as edge attributes. 


 The network import function cannot import node attributes - only edge attributes. To import node attributes from this table, please see the Attributes section of this manual. 


 Note (1): This data is taken from the \emph{A merged human interactome}
 datasets by Andrew Garrow, Yeyejide Adeleye and Guy Warner (Unilever, Safety and Environmental Assurance Center, 12 October 2006). Actual data files are available at \url{http://www.cytoscape.orghttp://cytoscape.org/cgi-bin/moin.cgi/Data}\_Sets/


 
\textbf{Basic Operations}


 To import network text/Excel tables, please follow these steps: 
\begin{enumerate}
\item 

 Select File $\rightarrow$ Import $\rightarrow$ Network from Table (Text/MS Excel)... 

\item Select a table file by clicking on the Select File button. 
\item Define the interaction parameters by specifying which columns of data contain the Source Interaction, Target Interaction, and Interaction Type. Setting the Interaction Type as Default Interaction will result in all interactions being given the value pp; this value can be modified in Advanced Options (below). 
\item (Optional) Define edge attribute columns, if applicable. Network table files can have edge attribute columns in addition to network data. \begin{itemize}
\item 

 Enable/Disable Attribute Column - By \emph{left}
-clicking on a column header in the preview table, you can enable/disable edge attributes. If the header is checked and entries are blue, the column will be imported as an edge attribute. For example, the table below shows that columns 1 through 3 will be used as network data, column 4 will not be imported, and columns 5 and 6 will be imported as edge attributes. 
\begin{center}
 
\includegraphics[wdith=\textwidth]{images/network_table_sample.png} 
\end{center}

\item 

 Change Attribute Name and Data Types - If you \emph{right}
-click on a column header in the preview table, you can modify the attribute name and data type. For more detail, see ``Modify Attribute Name/Type'' below. 

\end{itemize}
\item Click the Import button. 
\end{enumerate}

 \textbf{Import List of Nodes Without Edges}

 Table Import feature supports list of nodes without edges. If you select source column only, it creates a network without interactions. This feature is usufl with node expansion function available from some web service clients. Please read the section \emph{Importing Networks from External Database}
 for more detail. 

\textbf{Advanced Options}


 \includegraphics[wdith=\textwidth]{images/network_import_advanced.png} 


 You can select several options by checking the Show Text File Import Options checkbox. 
\begin{itemize}
\item Delimiter: You can select multiple delimiters for text tables. By default, Tab and Space are selected as delimiters. 
\item Preview Options: When you select a network table file, the first 100 entries will be displayed in the Preview panel. To display more entries, change the value for this option. If you want to show all entries in the file, select ``Show all entries in the file''. You will need to click the Reload button to update the Preview panel. 
\item Attribute Names \begin{itemize}
\item Transfer first line as attribute names: Selecting this option will cause all edge attribute columns to be named according to the first data entry in that column. 
\item Start Import Row: Set which row of the table to begin importing data from. For example, if you want to skip the first 3 rows in the file, set 4 for this option. 
\item Comment Line: Rows starting with this character will not be imported. This option can be used to skip comment lines in text files. 
\end{itemize}
\item Network Import Options: If the Interaction Type is set to Default Interaction, the value here will be used as the interaction type for all edges. 
\end{itemize}
 
\textbf{Modify Attribute Name/Type}
 \includegraphics[wdith=\textwidth]{images/network_table_attr_dialog1.png} 

 Attribute names and data types can be modified here. 
\begin{itemize}
\item Modify Attribute Name - just enter a new attribute name and click OK. 
\item Modify Attribute Data Type - The following attribute data types are supported: \begin{itemize}
\item String 
\item Boolean (True/False) 
\item Integer 
\item Floating Point 
\item List of (one of) String/Boolean/Integer/Floating Point 
\end{itemize}
\end{itemize}
 Cytoscape has a basic data type detection function that automatically suggests the attribute data type of a column according to its entries. This can be overridden by selecting the appropriate data type from the radio buttons provided. For lists, a global delimiter must be specified (i.e., all cells in the table must use the same delimiter). 

\section{Import Networks from Web Services}

 From version 2.6.0, Cytoscape has a new feature called \textbf{Web Service Client Manager}
. Users can access verious kinds of databases through this function. Please read \emph{\textbf{Importing Networks and Attributes from External Database}
}
 for more detail. 

\section{Edit a New Network}

 A new, empty network can also be created and nodes and edges manually added. To create an empty network, go to File $\rightarrow$ New $\rightarrow$ Network $\rightarrow$ Empty Network, and then manually add network components using the Editor in CytoPanel 1 (see the Editor chapter for more details). 
