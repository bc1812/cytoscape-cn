除了普通的节点和边属性数据,Cytoscape~还支持导入基因表达数据。基因表达数据的格式跟一般的属性文件的格式不一样,但是对于Cytoscape来说,最终对这些属性的处理方法却没有区别。基因表达数据(跟属性数据一样)在任何时候都可以导入,但(通常)都是在网络加载后导入一次。 

 \subsubsection{数据文件格式}

一个或多个实验所得到的基因表达比例(gene expression ratio)或称基因表达值(value)都是用文本文件存储的。比例结果来自于两个基因表达测试(实验组和对照组)的比较。在某些基因表达平台上,例如Affymetrix,可以直接测出基因表达值,而无需比较。文件包括一个头部,以及若干用空格或制表符分割的域,每一行都是一个基因,大概的格式如下:

 \begin{verbatim}
Identifier [CommonName] value1 value2 ... valueN [pval1 pval2 ... pvalN]
\end{verbatim}

方括号\[ \]表示该域是可选的。

第一个域用于识别该基因对应Cytoscape中的哪个节点。在最简单的情况下,这就是基因的名称,跟其在Cytoscape中的网络中的名称完全一致(大小写敏感)。此外,也可以是能识别节点的某些属性,例如商业微阵列的探针标识。

 The next field is an optional common name. It is not used by Cytoscape, and is provided strictly for the user's convenience. With this common name field, the input format is the same as for commonly-used expression data anaysis packages such as SAM (\url{http://www-stat.stanford.edu/~tibs/SAM/}). 

 The next set of columns represent expression values, one per experiment. These can be either absolute expression values or fold change ratios. Each experiment is identified by its experiment name, given in the first line. 

 Optionally, significance measures such as P values may be provided. These values, generated by many microarray data analysis packages, indicate where the level of gene expression or the fold change appears to be greater than random chance. If you are using significance measures, then your expression file should contain them in a second set of columns after the expression values. The column names for the expression significance measures need to match those of the expression values \textbf{exactly} . 

看一个例子,下面是Cytoscape的sampleData目录中galExpData.pvals文件的节选:

 \begin{verbatim}
GENE COMMON gal1RG gal4RG gal80R gal1RG gal4RG gal80R
YHR051W COX6 -0.034 0.111 -0.304 3.75720e-01 1.56240e-02 7.91340e-06
YHR124W NDT80 -0.090 0.007 -0.348 2.71460e-01 9.64330e-01 3.44760e-01
YKL181W PRS1 -0.167 -0.233 0.112 6.27120e-03 7.89400e-04 1.44060e-01
YGR072W UPF3 0.245 -0.471 0.787 4.10450e-04 7.51780e-04 1.37130e-05
\end{verbatim}

 This indicates that there is data for three experiments: gal1RG, gal4RG, and gal80R. These names appear two times in the header line: the first time gives the expression values, and the second gives the significance measures. For instance, the second line tells us that in Experiment gal1RG, the gene YHR051W has an expression value of -0.034 with significance measure 3.75720e-01. 

 Some variations on this basic format are recognized; see the formal file format specification below for more information. Expression data files commonly have the file extensions ``.mrna'' or ``.pvals'', and these file extensions are recognized by Cytoscape when browsing for data files. 
 
\subsubsection{基本过程}
 Load an expression attribute matrix file using File \^a†’ Import \^a†’ Attribute/Expression Matrix... to bring up the import window, or by specifying the filename using the -m option at the command line. If you use the command line input, you must enter your expression data by node ID. If you use the dialog box, then you can either load expression data by node ID (the default option), or you can select a node attribute to use in assigning your expression data to your Cytoscape nodes. If you do use a node attribute, then (1) the attribute should already be loaded, and (2) the node attribute value must match the first column in your matrix file. 
 
\subsubsection{实例}
下面针对sampleData/galFiltered.sif网络演示一下如何导入基因表达数据: 

\begin{enumerate}
\renewcommand{\labelenumi}{\textbf{Option \Alph{enumi}.}}
\item 点击File $rightarrow$ Import $rightarrow$ Attribute/Expression Matrix\ldots 。在结果窗口中,有一个名为``Please select an attribute or expression matrix file\ldots'',用Select按钮输入sampleData/galExpData.pvals。这个文件中的标识符和sampleData/galFiltered.sif网络数据中的标识符是一致的,所以不需要使用``Assign values to nodes using...''。下面是这个文件中的几行内容:
 \begin{verbatim}
GENE COMMON gal1RG gal4RG gal80R gal1RG gal4RG gal80R
YHR051W COX6 -0.034 0.111 -0.304 3.75720e-01 1.56240e-02 7.91340e-06
YHR124W NDT80 -0.090 0.007 -0.348 2.71460e-01 9.64330e-01 3.44760e-01
YKL181W PRS1 -0.167 -0.233 0.112 6.27120e-03 7.89400e-04 1.44060e-01
\end{verbatim}
\item 第1步:在加载网络后,再用File$\rightarrow$Import$\rightarrow$Node attributes\ldots加载sampleData/gal.probeset.na中的节点属性。下面是这个文件的部分内容:
 \begin{verbatim}
Probeset
YHR051W = probeset2
YHR124W = probeset3
YKL181W = probeset4
\end{verbatim}

第2步:在加载了节点属性文件后,选择基因表达数据文件sampleData.galExpPvals.probeset.pvals,下面是该文件的部分内容: 
 \begin{verbatim}
GENE COMMON gal1RG gal4RG gal80R gal1RG gal4RG gal80R
probeset2 COX6 -0.034 0.111 -0.304 3.75720e-01 1.56240e-02 7.91340e-06
probeset3 NDT80 -0.090 0.007 -0.348 2.71460e-01 9.64330e-01 3.44760e-01
probeset4 PRS1 -0.167 -0.233 0.112 6.27120e-03 7.89400e-04 1.44060e-01
\end{verbatim}
 After selecting this file, in the field labeled ``Assign values to nodes using...'', select Probeset. You will see that this loads exactly the same expression data as in Case 1, but provides extra flexibility in case the node name cannot be used as an identifier. 
 \end{enumerate}

\subsubsection{文件格式的详细说明(高级用户)}
 In all expression data files, any whitespace (spaces and/or tabs) is considered a delimiter between adjacent fields. Every line of text is either the header line or contains all the measurements for a particular gene. No name conversion is applied to expression data files. 

 The names given in the first column of the expression data file should match exactly the names used elsewhere (i.e. in SIF or GML files). 

 The first line is a header line with one of the following three header formats: 

 \begin{verbatim}
<text> <text> cond1 cond2 ... cond1 cond2 ... [NumSigConds]
<text> <text> cond1 cond2 ...
<tab><tab>RATIOS<tab><tab>...LAMBDAS
\end{verbatim}

 The first format specifies that both expression ratios and significance values are included in the file. The first two text tokens (in angled brackets) contain names for each gene, such as the formal and common gene names. The condX token set specifies the names of the experimental conditions; these columns will contain ratio values. This list of condition names must then be duplicated exactly, each spelled the same way and in the same order. Optionally, a final column with the title NumSigConds may be present. If present, this column will contain integer values indicating the number of conditions in which each gene had a statistically significant change according to some threshold. 

 The second format is similar to the first except that the duplicate column names are omitted, and there is no NumSigConds field. This format specifies data with ratios but no significance values. 

 The third format specifies an MTX header, which is a commonly used format. Two tab characters precede the RATIOS token. This token is followed by a number of tabs equal to the number of conditions, followed by the LAMBDAS token. This format specifies both ratios and significance values. 

 Each line after the first is a data line with the following format: 

 \begin{verbatim}
FormalGeneName CommonGeneName ratio1 ratio2 ... [lambda1 lambda2 ...] [numSigConds]

\end{verbatim}
 The first two tokens are gene names. The names in the first column are the keys used for node name lookup; these names should be the same as the names used elsewhere in Cytoscape (i.e. in the SIF, GML, or XGMML files). Traditionally in the gene expression microarray community, who defined these file formats, the first token is expected to be the formal name of the gene (in systems where there is a formal naming scheme for genes), while the second is expected to be a synonym for the gene commonly used by biologists, although Cytoscape does not make use of the common name column. The next columns contain floating point values for the ratios, followed by columns with the significance values if specified by the header line. The final column, if specified by the header line, should contain an integer giving the number of significant conditions for that gene. Missing values are not allowed and will confuse the parser. For example, using two consecutive tabs to indicate a missing value will not work; the parser will regard both tabs as a single delimiter and be unable to parse the line correctly. 

 Optionally, the last line of the file may be a special footer line with the following format: 

 \begin{verbatim}
NumSigGenes int1 int2 ...

\end{verbatim}

 This line specified the number of genes that were significantly differentially expressed in each condition. The first text token must be spelled exactly as shown; the rest of the line should contain one integer value for each experimental condition. 
